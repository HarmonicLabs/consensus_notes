\chapter{Consensus}

Consensus is the process of reaching a majority opinion by everyone 
involved in running the blockchain. An agreement must be made on which 
blocks to produce, which chain to adopt and to determine the single 
state of the network. The consensus protocol determines how individual 
nodes assess the current state of the ledger system and reach a 
consensus.

\vspace{0.2cm}

\noindent
Blockchains create consensus by allowing participants to bundle 
transactions that others have submitted to the system in blocks, and add 
them to their chain (sequence of blocks). Determining who is allowed to 
produce a block when, and what to do in case of conflicts, (such as two 
participants adding different blocks at the same point of the chain), is 
the purpose of the different consensus protocols.

\vspace{0.2cm}

\noindent
The protocol has three main responsibilities:
\begin{itemize}
    \item perform a leader check and decide if a block should be produced;
    \item handle chain selection;
    \item verify produced blocks.
\end{itemize}

\section{Ouroboros}

Cardano runs on the Ouroboros \gls{Proof-of-Stake} consensus protocol.

\vspace{0.2cm}

\noindent
Ouroboros divides time on Cardano into epochs where each epoch is divided 
into slots. A slot is a short period of time in which a block can be 
created. Grouping slots into epochs is central to adjusting the leader 
election process to the dynamically changing stake distribution.

\vspace{0.2cm}

\noindent
A slot leader is elected for each slot, who is responsible for adding a 
block to the chain and passing it to the next slot leader. To protect 
against adversarial attempts to subvert the protocol, each new slot leader 
is required to consider the last few blocks of the received chain as 
transient: only the chain that precedes the prespecified number of transient 
blocks is considered settled. This is also referred to as the settlement 
delay. Among other things, this means that a stakeholder can go offline and 
still be synced to the blockchain, so long as it's not for more than the 
settlement delay.

\vspace{0.2cm}

\noindent
Within the Ouroboros protocol, each network node stores a copy of the 
transaction mempool (where transactions are added if they are consistent 
with existing transactions) and the blockchain. The locally stored blockchain 
is replaced when the node becomes aware of an alternative, longer valid chain.

\vspace{0.2cm}

\subsection{Versions}

\subsubsection{Classic}

The first implementation of Ouroboros achieved three major milestones:
\begin{itemize}
    \item the foundation for an energy-efficient protocol to rival proof of work;
    \item the introduction of the mathematical framework to analyze proof of stake;
    \item the implementation of a novel incentive mechanism to reward participants 
        in a proof-of-stake setting.
\end{itemize}

\noindent
But what really set Ouroboros apart from other blockchain protocols (specifically, 
proof-of-stake protocols), was its ability to generate unbiased randomness in the 
protocol's leader selection algorithm, and the subsequent security assurances that 
provided. Randomness prevents the formation of patterns, which is critical for 
maintaining the protocol's security. Ouroboros was the first blockchain protocol 
developed with this type of rigorous security analysis.

\vspace{0.5cm}

\subsubsection{BFT}

Ouroboros Byzantine Fault Tolerance (BFT) was the protocol's second implementation, 
used during the Byron update (transition from the old Cardano codebase to the 
new). The second instance of the protocol prepared Cardano for the decentralization 
that came with the Shelley release.

\vspace{0.2cm}

\noindent
Ouroboros BFT enabled synchronous communication between a network of federated 
servers (the blockchain) providing ledger consensus in a simpler and more 
deterministic manner.

\vspace{0.5cm}

\subsubsection{Praos}

Ouroboros Praos introduced substantial security and scalability improvements to 
the Ouroboros Classic implementation. Praos processes transaction blocks by dividing 
chains into slots, which are aggregated into epochs. But unlike Ouroboros Classic, 
Praos is analyzed in a semi-synchronous setting and is secure against adaptive 
attackers, using private-leader selection and forward-secure, key-evolving signatures 
to ensure that a strong adversary cannot predict the next slot leader and launch 
a focused attack (such as a DDoS attack).

\vspace{0.5cm}

\subsubsection{Genesis}

The fourth iteration of Ouroboros, Genesis, further improves upon Ouroboros 
Praos by adding a novel chain selection rule that enables parties to bootstrap 
from a genesis block without the need for trusted checkpoints or assumptions 
about past availability. The Genesis paper also provides proof of the protocol's 
Universal Composability, which demonstrates that the protocol can be composed 
with other protocols in arbitrary configurations in a real-world setting, without 
losing its security properties.

\vspace{0.5cm}

\subsubsection{Crypsinous}

Ouroboros Crypsinous equips Genesis with privacy-preserving properties. It is 
the first formally analyzed privacy-preserving proof-of-stake blockchain protocol, 
which achieves security against adaptive attacks while maintaining strong 
privacy guarantees by introducing a new coin evolution technique relying on 
Snarks and key-private forward-secure encryption. Crypsinous isn't currently 
planned to be implemented on Cardano, but it can be used by other chains for 
increased privacy-preserving settings.

\vspace{0.5cm}

\subsubsection{Chronos}

Chronos achieves two goals: 
\begin{itemize}
    \item it shows how blockchain protocols can synchronize clocks securely via 
        a novel time synchronization mechanism and thereby become independent of 
        external time services;
    \item it provides a cryptographically secure source of time to other protocols.
\end{itemize}
In short, Chronos makes the ledger more resistant to attacks that target time 
information.

\vspace{0.2cm}

\noindent
From an application point of view, Chronos can dramatically boost the resilience 
of critical telecommunications, transport, and other IT infrastructures that require 
the synchronization of local time to a unified network clock that has no single 
point of failure.