\chapter{Cardano}

Cardano is a \gls{Proof-of-Stake} blockchain platform: the first 
to be founded on peer-reviewed research and developed through 
evidence-based methods. It combines pioneering technologies 
to provide unparalleled security and sustainability to decentralized 
applications, systems, and societies.

\section{Cardano Development Phases}

The five phases of Cardano represent a gradual and well-planned development 
path aimed at building a secure, scalable, and sustainable blockchain. 
Cardano's roadmap is unique in its methodical and research-based approach, 
making the project one of the most ambitious and promising platforms in the 
industry. As each phase progresses, Cardano moves closer to its goal of 
becoming a leading global blockchain platform capable of supporting a wide 
range of decentralized applications and services.

\vspace{0.2cm}

\subsection{Byron (Foundation)}

\subsubsection*{Main goals}
\begin{itemize}
    \item \textbf{Launch of the Mainnet}: the Byron phase marks the launch of 
        Cardano's mainnet in September 2017, making Cardano available 
        to the public;
    \item \textbf{Creation of the Daedalus Wallet}: during this phase, the 
        official Cardano desktop wallet, Daedalus, was released. It was 
        designed to be secure and user-friendly;

    \item \textbf{Implementation of the Blockchain}: this phase saw the 
        creation and implementation of the blockchain based on Ouroboros, 
        Cardano's Proof-of-Stake consensus protocol;

    \item \textbf{Support for ADA Cryptocurrency}: the Byron phase introduced 
        ADA, Cardano's native cryptocurrency, which can be bought, sold, and 
        stored in the Daedalus wallet.
\end{itemize}

\subsubsection*{Technical features}
\begin{itemize}
    \item \textbf{Proof-of-Stake}: introduction of the Proof-of-Stake (PoS) 
        consensus mechanism through Ouroboros;
    \item \textbf{Blockchain Explorer}: a tool that allows users to check 
        transactions on the Cardano network.
\end{itemize}

\subsubsection*{Outcomes}
The Byron phase established the foundation of the Cardano blockchain, ensuring 
the network was secure, reliable, and capable of supporting further developments.

\vspace{0.3cm}

\subsection{Shelley (Decentralization)}

\subsubsection*{Main goals}
\begin{itemize}
    \item \textbf{Decentralization}: Shelley’s primary objective was to decentralize 
        the network. While Byron laid the groundwork, Shelley introduced mechanisms to 
        transition to decentralized governance;
    \item \textbf{Staking and Stake Pools}: introduction of staking features, allowing 
        users to participate in the network and earn rewards.
\end{itemize}

\subsubsection*{Technical features}
\begin{itemize}
    \item \textbf{Incentive System}: design of an incentive system to reward 
        users for maintaining and securing the network;
    \item \textbf{Delegation}: ability for users to delegate their stake to staking 
        pools to earn rewards;
    \item \textbf{Increase in Independent Nodes}: growth in the number of independent 
        nodes, enhancing the network's security and resilience.
\end{itemize}

\subsubsection*{Outcomes}
Shelley transitioned Cardano from a federated platform to a decentralized one, 
increasing community participation and establishing a distributed governance model.

\vspace{0.3cm}

\subsection{Goguen (Smart Contracts)}

\subsubsection*{Main goals}
\begin{itemize}
    \item \textbf{Smart Contracts}: the Goguen phase introduced support for \gls{Smart 
        Contracts}, enabling developers to build decentralized applications (DApps) 
        on the Cardano platform;
    \item \textbf{Compatibility with other Blockchains}: improved interoperability 
        with other blockchains and legacy systems.
\end{itemize}

\subsubsection*{Technical features}
\begin{itemize}
    \item \textbf{Plutus}: a new programming language for writing secure and reliable 
        smart contracts. Plutus allows for the development of more complex applications 
        on Cardano;
    \item \textbf{Marlowe}: another programming language specifically designed for 
        financial contracts, enabling the creation of smart contracts even by users 
        without programming experience;
    \item \textbf{Support for Native Tokens}: introduction of support for creating 
        and managing native tokens on the Cardano blockchain, allowing for asset 
        tokenization.
\end{itemize}

\subsubsection*{Outcomes}
The Goguen phase made Cardano a versatile and competitive platform for developing DApps 
and smart contracts, expanding the network's use cases.

\vspace{0.3cm}

\subsection{Basho (Scaling)}

\subsubsection*{Main goals}
\begin{itemize}
    \item \textbf{Scalability}: improve the network's scalability to handle more 
        transactions per second, making Cardano suitable for large-scale adoption;
    \item \textbf{Performance Optimization}: optimize network performance, reducing 
        latency, and improving overall efficiency.
\end{itemize}

\subsubsection*{Technical features}
\begin{itemize}
    \item \textbf{Sidechains}: introduction of sidechains, which expand network capacity 
        without compromising the security and integrity of the main blockchain;
    \item \textbf{Protocol Improvements}: updates and optimizations to the Ouroboros 
        consensus protocol to enhance performance.
\end{itemize}

\subsubsection*{Outcomes}
The Basho phase laid the groundwork for a more efficient and scalable network, ensuring 
that Cardano can support a wide range of applications and an increasing number of users.

\vspace{0.3cm}

\subsection{Voltaire (Governance)}

\subsubsection*{Main goals}
\begin{itemize}
    \item \textbf{Decentralized Governance}: implement a decentralized governance system, 
        allowing the community to make decisions regarding the network's future developments;
    \item \textbf{Sustainability}: create a sustainable funding system for the platform's 
        maintenance and development.
\end{itemize}

\subsubsection*{Technical features}
\begin{itemize}
    \item \textbf{Treasury System}: introduction of a treasury system that collects funds 
        to finance development projects proposed by the community;
    \item \textbf{Voting and Improvement Proposals}: implementation of a voting mechanism 
        that enables stakeholders to propose and vote on improvements and changes to the 
        network.
\end{itemize}

\subsubsection*{Outcomes}
Voltaire transforms Cardano into a fully autonomous and self-governing platform, allowing 
the community to directly influence the network's future.

\vspace{0.5cm}

\section{Ledger}

The Cardano Ledger refers to the underlying structure and technology that supports 
Cardano's blockchain. It represents the complete system that records all transactions and 
smart contracts executed on the Cardano network, ensuring their security, transparency, 
and immutability.

\vspace{0.2cm}

\subsection{Eras}

There are several eras within the evolution of Cardano. Each era refers to the rules of 
the ledger. For example, what transaction types and what data is stored in the ledger, 
or the validity and meaning of the transactions.

\vspace{0.5cm}

\begin{tabular}{ ||p{1.25cm}||p{1.5cm}|p{1.5cm}|p{2cm}|p{2cm}|p{3cm}||  }
    \hline
        DATE & PHASE & ERA & LEDGER PROTOCOL & PROTOCOL VERSION & CONSENSUM MECHANISM \\
    \hline\hline
    2017/09 & Byron & Byron & - & 0.0 & Ouroboros Classic \\
    \hline
    2020/02 & Byron & Byron & - & 1.0 & Ouroboros BFT \\
    \hline
    2020/07 & Shelley & Shelley & TPraos & 2.0 & Ouroboros Praos \\
    \hline
    2020/12 & Goguen & Allegra & TPraos & 3.0 & Ouroboros Praos \\
    \hline
    2021/03 & Goguen & Mary & TPraos & 4.0 & Ouroboros Praos \\
    \hline
    2021/09 & Goguen & Alonzo & TPraos & 5.0 & Ouroboros Praos \\
    \hline
    2021/10 & Goguen & Alonzo & TPraos & 6.0 & Ouroboros Praos \\
    \hline
    2022/09 & Goguen & Babbage & Praos & 7.0 & Ouroboros Praos \\
    \hline
    2023/02 & Goguen & Babbage & Praos & 8.0 & Ouroboros Praos \\
    \hline
\end{tabular}

\vspace{0.5cm}

\subsubsection{Byron and Shelley eras}

The evolution of the Cardano mainnet began with the Byron ledger rules. The mainnet 
underwent a \gls{Hard Fork} in late July 2020 to switch from the Byron rules to the Shelley 
ledger rules. It was a full reimplementation of Cardano, which enabled two fundamental 
changes: the support for multiple sets of ledger rules, and the management of the hard 
fork process of switching from one set of rules to the next. In other words, the new 
implementation could support both the Byron rules and the Shelley rules, which meant that, 
when deployed to the mainnet in early 2020, the implementation was fully compatible with 
the Byron rules. This allowed for a smooth transition from the old to the new implementation. 
Once all Cardano users had upgraded their nodes to the new implementation, it became 
possible to invoke the hard fork combinator event and switch to the Shelley rules.

\vspace{0.5cm}

\subsubsection{Allegra, Mary, and Alonzo eras}

Allegra, Mary, and Alonzo eras are all part of the Goguen development phase.

\vspace{0.2cm}

\noindent
Starting with Goguen, the ledger team introduced the notion of era into the ledger code. 
Shelley ledger rules then became 'the Shelley era'.

\vspace{0.2cm}

\noindent
Because Goguen features were implemented in steps, each set of functionality was introduced 
with a different hard fork, hence there were several ledger eras:
\begin{itemize}
    \item Allegra: introduced token locking support;
    \item Mary: brought native tokens and multi-asset functionality to Cardano
    \item Alonzo: introduced smart contract support.
\end{itemize}

\vspace{0.2cm}

\subsubsection{Bubbage era}

The Babbage ledger era introduced such features as inline datums, reference scripts, and 
reference inputs.

\vspace{0.5cm}

\section{Nodes}

The Cardano node is the top-level component within the network. Network nodes connect to 
each other within the networking layer, which is the driving force for delivering 
information exchange requirements. This includes new block diffusion and transaction 
information for establishing a better data flow. Cardano nodes maintain connections 
with peers that have been chosen via a custom peer-selection process. By running a 
Cardano node, you are participating in and contributing to the network.

\vspace{0.2cm}

\noindent
Stake pools use the Cardano node to validate how the pool interacts with the network 
and are responsible for transaction processing and block production. They act as reliable 
server nodes that hold and maintain the combined stake of various stakeholders in a 
single entity.

\vspace{0.2cm}

\subsection{Stake Pools}

A stake pool is a reliable server node that focuses on ledger 
maintenance and holds the combined resources - the 'stake' - of 
various stakeholders in a single entity. Stake pools are responsible 
for processing transactions that will be placed in the ledger, as 
well as producing new blocks. Stake pools are at the core of Ouroboros, 
Cardano's proof-of-stake protocol.

\vspace{0.2cm}

\noindent
To be secure, Ouroboros requires a good number of stakeholders to be 
online and maintain sufficiently good network connectivity at any given 
time. This is why Ouroboros relies on stake pools, entities that are 
committed to running the protocol 24/7, on behalf of the contributing 
stakeholders that hold ADA. The idea is that these resource holders can 
bring their resources (their stake) together and form a pool, where 
typically one holder is the operator of the pool and the rest are 
delegators.

\vspace{0.5cm}

\section{Blocks}

The goal of blockchain technology is the production of an 
independently-verifiable and cryptographically-linked chain of records 
(blocks). A network of block producers works to collectively advance the 
blockchain. A consensus protocol (Ouroboros) provides transparency and 
decides which candidate blocks should be used to extend the chain.

\vspace{0.2cm}

\noindent
Submitted valid transactions might be included in any new block. 
A block is cryptographically signed by its producer and linked to the 
previous block in the chain. This makes it impossible to delete 
transactions from a block, alter the order of the blocks, remove a block 
from the chain or insert a new block into the chain without alerting all 
the network participants. This ensures the integrity and transparency of 
the blockchain expansion.

\vspace{0.2cm}

\subsection{Slots and Transactions}

The Cardano blockchain uses the Ouroboros Praos protocol to facilitate 
consensus on the chain. Ouroboros Praos divides time into epochs. Each 
Cardano epoch consists of a number of slots, where each slot lasts for 
one second. A Cardano epoch currently includes 432,000 slots (5 days). 
In any slot, zero or more block-producing nodes might be nominated to 
be the slot leader. On average, one node is expected to be nominated 
every 20 seconds, for a total of 21,600 nominations per epoch. If 
randomly elected slot leaders produce blocks, one of them will be added 
to the chain. Other candidate blocks will be discarded.

\vspace{0.5cm}

\subsubsection{Slot leader election}

The Cardano network consists of a number of stake pools that control 
the aggregated stake of their owners and other stakeholders, also known 
as delegators. Slot leaders are randomly elected from among the stake 
pools. The more stake a pool controls, the greater the chance it has 
of being elected as a slot leader to produce a new block that is accepted 
into the blockchain. This is the basic concept of proof of stake (PoS). 
To maintain a level playing field, and prevent a situation where a small 
number of very large pools control the majority of stake, Cardano has an 
incentive system that discourages delegation to pools that already control 
a large portion of the total stake.

\vspace{0.5cm}

\subsubsection{Transaction validation}

When validating a transaction, a slot leader needs to ensure that the 
sender has included enough funds to pay for that transaction and must 
also ensure that the transaction's parameters are met. Assuming that 
the transaction meets all these requirements, the slot leader will 
record it as a part of a new block, which will then be connected to 
other blocks in the chain.